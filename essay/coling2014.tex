\documentclass[11pt]{article}
\usepackage{coling2014}
\usepackage{times}
\usepackage{url}
\usepackage{latexsym}
\usepackage{apacite}
\usepackage[colorinlistoftodos]{todonotes}

%\setlength\titlebox{5cm}

% You can expand the titlebox if you need extra space
% to show all the authors. Please do not make the titlebox
% smaller than 5cm (the original size); we will check this
% in the camera-ready version and ask you to change it back.


\title{Ethics in computational decision-making}

\author{First Author \\
  Affiliation / Address line 1 \\
  Affiliation / Address line 2 \\
  Affiliation / Address line 3 \\
  {\tt email@domain} \\\And
  Second Author \\
  Affiliation / Address line 1 \\
  Affiliation / Address line 2 \\
  Affiliation / Address line 3 \\
  {\tt email@domain} \\}

\date{}

\begin{document}
\maketitle
\begin{abstract}
  In this essay we examine the ethical aspects of computational decision-making.
\end{abstract}

\section{Introduction}
Computers help us make decisions every day. They recommend products for us to
buy, they correct our spelling errors and they suggest whom we should befriend
on social network sites like Facebook. But as computers become more and more
powerful and research in artificial intelligence progresses, the idea of letting
computers make the decisions that have typically been made by humans becomes
more interesting. There are already some applications for decision-making
computers, such as the bots trading on the stock market. In their paper ‘The Ethics of Artificial Intelligence’ \citeyear{bostrom2013ethics}, \citeauthor{bostrom2013ethics} argued that increasingly complex decision-making algorithms are both inevitable and desirable. 

\paragraph{But when computer AIs will make decisions that more directly affect humans}, a series of questions
arise: Will the AIs make ethical decisions? Who is to blame for a bad decision
made by an AI? Can we trust the decision-making process? This essay will argue 
for both the positive and negative aspects regarding the ethics of a computer 
based decision and present our point of view as computer science students. 
This essay will argue for both the positive and negative aspects regarding the ethics of a computer based decision and present our point of view as computer science students. 

\section{In what areas could computer-decisions have ethical effects?}
There are a number of applications where AI algorithms may take important decisions that directly
affect humans. For example the possibility of
a bank using a machine learning algorithm to approve or reject mortgage
applications \cite{bostrom2013ethics}. 

\paragraph{There is research into algorithms used to predict court decisions}
\cite{martin2004competing}, with algorithms predicting as much
as 70\% of the outcomes correctly \cite{Kravets2014Court} it is
conceivable to think of an algorithm taking the actual decision of court
outcomes some day in the future. When making big decisions like these, it
becomes important for the algorithm to take in ethical aspects.

\paragraph{\citeA{Chatfield2014automate} states that medical triage is a field in which automation and algorithms already play a considerable part}. Triage is how to priorities wounded or sick patience. It could be who to treat first in an emergency situation. Should one help the most critical injured, help a child. If two lightly injured patients could be helped simultaneously, is it better than helping one more critically injured? In these areas Artificial Intelligence decision making could be applied.

\paragraph{\citeauthor{Chatfield2014automate} also writes about the car industry.} The evolution of cars has progressed nearly one hundred years. For each year the car gets safer and safer. But still millions of peoples dies in traffic each year. There is one part which technology can improve, the human being driving the car. This can be solved by autonomous cars. \citeauthor{elonmuskTweet} believes that when autonomous cars are safer than self-driven cars, the public may outlaw the self-driven ones. But autonomous cars is also a question of ethical decision-making. For example some school children are crossing the road and an accident is unavoidable, should the car swerve risking the drivers life or break risking the childrens life? A ethical question.

\paragraph{Even in the late stages of the human life} one could encounter machine based decision-making. In their article “Robot Be Good: A Call for Ethical Autonomous
Machines”, \citeauthor{anderson2010robot} describe a robot which
helps the elderly in various way, for example by reminding them to take their
medicine. In such a situation, the robot would have to take into consideration
the autonomy of the person in question, who maybe doesn't want to take their
medicine, as well as predicting the outcome if the person does not take their
medicine. Self-driving cars is also a type of artificial intelligence that will
have to take ethical decisions.

All cases above would benefit if one could make decisions based on pure facts. Which is not possible for a human
since we always will make subjective normative assessment. But computers can base decisions on pure facts and in all areas where this is desired computer based decision making will shine, as long as they are ethical right.

\section{Responsibility}
Imagine being arrested for a crime you did not commit.
Instead of the ordinary human judge or jury, the decision-making entity is an AI
algorithm. Despite your innocence, the algorithm comes to the conclusion that
you are guilty, and so the outcome of the trial puts you in jail.

A situation like the one described above would probably feel like a kafkaesque nightmare,
where the judgment of the AI appears like the judgment of a god. Who would be to
blame for an unfair decision from an AI? This is an important problem with
computational decision-making. Would the owner of the algorithm, e.g. a company
or as in this situation a court of law, be to blame? Would the company that
delivered the algorithm be responsible, or maybe the individual programmers who
worked on the project? Or is there maybe no one but the AI itself to blame, a
prospect that may seem silly today, but may be more reasonable in a future where
AIs approach sentience in their intelligence.

\citeauthor{mcfarland2014mind} presents a similar case in their report "Mind the Gap: Can Developers of Autonomous Weapons Systems Be Liable for War Crimes". They argue that the criminal laws of today is stated in a way that they have a need for personal accountability. That a victim of a crime needs to see the guilty part in remorse and paying for it in some way. They argues that even if a programmer takes responsibility for their algorithms, a victim still could see the company or even the robot guilty. These raises concerns.

Our opinion in the matter is to have some insight in the decision-making process to be able to find the guilty party. 

% In this mess of trying to decide
% whom to blame, it could be easy for a company, or a court of law, to avoid
% taking responsibility. A decision-making algorithm can be used as a scapegoat,
% in cases where traditionally the person who had made the decision would have
% been responsible. This is a question which is important to discuss, and we
% believe that there needs to be policy handling the matter.



\section{What is ethics and how big must the ethical scoop be}
What constitutes ethical behavior might differ a
lot between different situations, applications, cultures and even over time. 
It is not many decades ago homosexuality was consider a disease and slavery was legit.
We believe that, despite the hard problem of encoding ethical behavior,
ultimately, an AI should be able to make better and more fair decisions than a
human being typically can. While a human can easily take in many aspects into
its decision-making and weighing complex considerations, a human is still very
controlled by its emotions. A human can be racist or homophobic, a human may
personally know the people involved in the decision, a human can simply have a
bad first impression of someone. An AI, on the other hand, can be encoded to not
take in certain aspects such as race and sexual orientation into its decision-
making. 

\todo{\citeauthor{moor2006nature} 4 ethical views, write about....}
% It is obvious that ethical behavior has to be encoded into these AI
% applications. How does one do this? Typical for an AI algorithm is its ability
% to deal with novel situations, situations that there is no predetermined way how
% to deal with. This makes it harder to program an AI to behave ethically; one
% cannot simply tell an AI how to act in all possible cases, but the AI would have
% to reason ethically. (fortfarande: hur?)

\section{How important is transparency in a computer made decision}
An important aspect of decision-making algorithms is transparency. We believe that to ensure the
fairness and correctness of a decision-making algorithm, it is important for
both the algorithm and its decision-making process to be transparent. Algorithms
that make decisions in for example a court of law should be published under an
open-source license, for anyone to study and analyze. But even with open-source
code, it can be hard to see why a certain decision was taken when it comes to AI
algorithms. Therefore, a trace of the reasoning leading up to a decision should
also be published together with the decision. Through transparency, we believe
the god-like judgment of an AI could feel more human and understandable.
Problem is, making an algorithm transparent is not always as easy as one could
think. Methods such as machine learning and artificial neural networks are great
for using training data to come up with new solutions, but exactly how the new
solutions are produced is often hard to decipher. One could publish the weights
used in an artificial neural network, or the distributions used in a machine-
learning algorithm, without this data making much sense. \todo{Nick Boström prater
om detta i The Ethics of Artificial Intelligence, bör nog skrivas om lite så
det blir en tydligare referens.} Even algorithms that are easier to understand
would for most people seem impossible to understand. Computer illiteracy is a
problem as it is today, but if computers also made important decisions in
society, than computer scientists and programmers would be part of a new upper
class in society. \todo{kanske lite väl luddigt skrivet} If computers should be
allowed to make important decisions in society, the decision-making process
should not only be transparent, but also easily understandable to non-computer
scientists.

\todo{Make examples of when one of us would be satisfied when...Denied to enter a nightclub by a computer, denied bank loan, not being medically prioritized in an accident.}


\subsection{How much transparency must be provided to be able to trust the outcome}

\section{Machines decision can never discriminate someone... or could it?}
Humans can not in retreat to a world of pure facts, or lace subjective normative assessment. Computers can!
Our thoughts of how a computer could be used in cases where discrimination is common. For example nightclub bouncer. Also argue about if a machine is not fully ethical agent, he can discriminate due to the

\section{When it comes to computer-made decisions does the end justifies the means}
Discuss the trolley problem. Give reasons from Consequentialism and Deontology.
What do we prefer? Use Automated ethics article.

\section{Our most humble conclusion}
Despite the difficulty of encoding ethical behavior in AIs,
assigning responsibility for the decisions taken and making the algorithms
transparent and understandable, we believe the computational decision-making
will prove very useful in the future. The prospect of letting unbiased
algorithms make important decisions instead of subjective humans is exciting. By
overcoming the aforementioned challenges, with the help of artificial
intelligence we may one day create a society which is more fair.

% \section*{Acknowledgements}

% The acknowledgements should go immediately before the references.  Do
% not number the acknowledgements section. Do not include this section
% when submitting your paper for review.

\bibliographystyle{apacite}
\bibliography{references}

\end{document}




